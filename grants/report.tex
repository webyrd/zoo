% Created 2015-07-01 Wed 07:26
\documentclass[11pt]{article}
\usepackage[utf8]{inputenc}
\usepackage[T1]{fontenc}
\usepackage{fixltx2e}
\usepackage{graphicx}
\usepackage{longtable}
\usepackage{float}
\usepackage{wrapfig}
\usepackage{soul}
\usepackage{textcomp}
\usepackage{marvosym}
\usepackage{wasysym}
\usepackage{latexsym}
\usepackage{amssymb}
\usepackage{hyperref}
\tolerance=1000
\providecommand{\alert}[1]{\textbf{#1}}

\title{SIGCSE Special Project Grant Report:\\Creating a Polished CS Education Zoo to Share CS Educators' Ideas}
\author{Steven Wolfman, Ph.D.\\Prof. of Teaching, UBC CS\\wolf@cs.ubc.ca}
\date{\today}
\hypersetup{
  pdfkeywords={},
  pdfsubject={},
  pdfcreator={Emacs Org-mode version 7.9.3f}}

\begin{document}

\maketitle

In January of 2015, the \href{http://webyrd.net/zoo.html}{CS Education Zoo} received a \$2700 SIGCSE
Special Project Grant to improve the polish and accessibility of the
Zoo. The Zoo is an ongoing series of hour-long video blogs led by Will
Byrd and Steven Wolfman in which we (the hosts) discuss CS Education
with guests from various positions and backgrounds.  Guests include
university CS educators in various positions, Computer Scientists in
industry positions with particular insight on CS education, CS
education researchers, and non-traditional CS educators.

Using the grant, we hired Piam Kiarostami---an alum of one of our
institutions, the University of British Columbia---who has:
contributed to both the ``normal course'' of interviews for the zoo,
spearheaded a new format (``Creature Features'') that creates excitement
about the Zoo on social media and identifies potential future guests,
indexed and tagged ``highlights'' from our existing Zoo interviews,
commenced a collaboration with Colleen Lewis and csteachingtips.org to
publicize these highlights, and prototyped improvements to the Zoo's
web presence.

Through this process, our audience for the Zoo has steadily
grown. Although some of our episodes still have around 200 views, our
most popular episodes have many hundred or thousands of views, with
Mark Guzdial's episode being the most popular at over 3,000 views,
compared to 225 views when we submitted the grant.

Our new ``Creature Features'' were a series of extremely brief
interviews of professionals interested in CS Education, filmed at
SIGCSE 2015. Piam worked with us to solicit, record, edit, and
publicize via Twitter roughly 20 short interviews. Of these, 11 are
now edited and linked as a ``Special Episode'' from the CS Education Zoo
site and include particularly interesting interviews like \href{https://youtu.be/NmkwNa493Y4}{Gerald Weiss on the end of the CS polymath}, \href{https://youtu.be/cosUyCwQA8w}{Brian Dorn on K-12 CS teaching}, and
\href{http://youtu.be/NGcT3ClAygo}{Tony Luckett on diversity in CS education}. We still have 3 usable
interviews left in editing; one of these, Ben Shapiro, is a good
example of someone that we identified from the Creature Features as an
excellent guest for the show. We're currently negotiating with him for
a time to join us this summer. We've posted the \href{https://goo.gl/B268dl}{full set of Creature Feature videos} online.

Springboarding off the Creature Features, Piam has established a two
pronged social media strategy for the Zoo:
\begin{enumerate}
\item Tweet highlights of interviews to increase awareness (and garner
   retweets).
\item Increase our social media presence at events like SIGCSE by
   tweeting links to quick 1 minute interviews with Hashtags to
   interested parties. This in turn builds momentum through retweets
   from the interviewees, their institutions, and their social
   network.
\end{enumerate}
Piam's knowledge of social media has led to particularly important
(but small) details including institution names in the tweets to
create more social networking momentum from the second piece of this
strategy.

Piam has also worked through all of the Zoo episodes to date to pull
out show ``highlights''. The highlights are brief cuts from the longer
interviews, each tagged with its topic. There are 62 of these ranging
in length from 1--6 minutes. Some particularly compelling examples
include: Mark Guzdial discussing what we called \href{https://youtu.be/ZpxxwZ9f_bo}{teaching ``worst'' practices}, mistakes that CS educators make; Colleen Lewis talking
about (and enthusiastically demonstrating) \href{https://youtu.be/3BVAaGWHD4I}{teaching arrays with Dixie cups}; and Rebecca Bates explains from the student, teacher, and
industry perspectives \href{https://youtu.be/f_0lbxv59iY}{what makes a good project course}. We're working
with Colleen Lewis in the hopes of adding these to the stream of
teaching tips at \href{http://www.csteachingtips.org}{csteachingtips.org} and tweeted to their 1000+
followers. We've posted the \href{https://goo.gl/jBrMBf}{full set of highlight videos}
online.

Although our website remains fairly primitive, Piam has added the
creature features and highlights to the existing page. Furthermore, he
has prototyped for us an alternate system for managing the site
directly via GitHub. (Presently, the site sits in a repository but is
separately pushed to the web.) The site has only placeholder content,
but is linked at \href{http://piam.github.io/zoo/}{http://piam.github.io/zoo/}, and we have instructions
on how to transition to this setup if we decide it's superior to our
existing arrangement.

Piam also centralized our video management and editing into a CS
Education Zoo Google account. He explored several other video editing
tools---including both free and trials of expensive paid options---but
concluded that the cloud-based Google tools were sufficiently powerful
to justify taking advantage of their much greater accessibility than
other dedicated video editing tools.

Our timeline on the project continues for two more months;
furthermore, Steve's teaching commitment to an early-summer intense
course interfered with our summer interview schedule. So, we hope to
accomplish more still---including recording and highlighting more
episodes of the zoo---and of course to continue the Zoo into the
future.

\end{document}
